%GLOSARIO

\newglossaryentry{JSX}
{
	name=JSX,
	description={es una extensión de la sintaxis del lenguaje JavaScript que proporciona una forma de estructurar la representación de componentes utilizando una sintaxis familiar para muchos desarrolladores. Es similar en apariencia a HTML} 
}

\newglossaryentry{XML}
{
	name=XML,
	description={siglas en inglés de eXtensible Markup Language, traducido como "Lenguaje de Marcado Extensible" o "Lenguaje de Marcas Extensible", es un metalenguaje que permite definir lenguajes de marcas desarrollado por el World Wide Web Consortium (W3C) utilizado para almacenar datos en forma legible. Proviene del lenguaje SGML y permite definir la gramática de lenguajes específicos (de la misma manera que HTML es a su vez un lenguaje definido por SGML) para estructurar documentos grandes. A diferencia de otros lenguajes, XML da soporte a bases de datos, siendo útil cuando varias aplicaciones deben comunicarse entre sí o integrar información} 
}

\newglossaryentry{API}
{
	name=API,
	description={una interfaz de programación de aplicaciones (API) es un conjunto particular de reglas (``código'') y las especificaciones que los programas de software pueden seguir para comunicarse entre sí} 
}
\newglossaryentry{borde}
{
	name=borde,
	description={Variaciones fuertes de la intensidad que	corresponden a las fronteras de los objetos visualizados} , plural= bordes
}


\newglossaryentry{chaar}
{
	name=cascada de Haar,
	description={El clasificador Haar es un método desarrollado por Viola y Jones y es una versión del algoritmo Adaboost. Es un clasificador basado en árboles de decisión con entrenamiento supervisado.} ,plural=cascadas de Haar
}
\newglossaryentry{roi}
{
	name=ROI,
	description={\textit{Region Of Interest} Una Región de interés es un subconjunto de una imagen (\gls{2D}) que es usado para un propósito específico} ,plural=ROIs
}
\newglossaryentry{arruga}
{
	name=arruga,
	description={Pliegue que se hace en la piel, ordinariamente por efecto de la edad} ,plural=arrugas
}
\newglossaryentry{mockups}
{
	name=mockups,
	description={Un wireframe o mockup básicamente es un boceto básico y de baja calidad del desarrollo de una página web o el diseño de una interfaz} 
}

\newglossaryentry{tipadodin}
{
	name=tipado dinámico,
	description={A diferencia de lenguajes como C, en lenguajes como \gls{python} a asignación de tipos a las variables se hace en tiempo de ejecución} 
}



\newglossaryentry{degeneracion}
{
	name=degeneración,
	description={facial, se entiende por degeneración cualquier herida, laceración o magulladura} 
}


\newglossaryentry{criminalistica}
{
	name=criminalística,
	description={es una disciplina que aplica fundamentalmente los conocimientos, métodos y técnicas de investigación de las ciencias naturales, en el examen del material sensible significativo relacionado con un presunto hecho delictuoso con el fin de determinar en auxilio de los órganos encargados de administrar justicia} 
}
\newglossaryentry{gui}
{
	name=interfaz de usuario,
	description={la interfaz de usuario es el medio con que el usuario puede comunicarse con una máquina, un equipo o una computadora, y comprende todos los puntos de contacto entre el usuario y el equipo} 
}


\newglossaryentry{2D}
{
	name=2D,
	description={Objeto cuyos componentes pueden ser ubicados mediante una 2-tupla $(x,y)$} 
}

\newglossaryentry{multiplataforma}
{
	name=multiplataforma,
	description={Que puede ser utilizado por distintos sistemas o entornos} 
}

\newglossaryentry{estereosc}
{
	name=imágenes estereoscópicas,
	description={La estereoscopía es cualquier técnica capaz de recoger información visual tridimensional y/o crear la ilusión de profundidad mediante una imagen estereográfica, un estereograma, o una imagen \gls{3D} (tridimensional)} 
}
\newglossaryentry{python}
{
	name=python,
	description={Python es un lenguaje de programación interpretado cuya filosofía hace hincapié en una sintaxis que favorezca un código legible. Se trata de un lenguaje de programación multiparadigma, ya que soporta orientación a objetos, programación imperativa y, en menor medida, programación funcional. Es un lenguaje interpretado, usa tipado dinámico y es multiplataforma} 
} 
\newglossaryentry{bash}
{
	name=bash,
	description={Bash (Bourne again shell) es un programa informático cuya función consiste en interpretar órdenes. Está basado en la shell de Unix y es compatible con POSIX.} 
} 

\newacronym[see={[Glosario:]{AGs}}]{AGs}{AGs}{Algoritmos genéticos \textit{(Algoritmos Genéticos)} \glsadd{AGs}}
\newacronym[see={[Glosario:]{APs}}]{APs}{APs}{Algoritmos de predicción \textit{(Algoritmos de predicción)} \glsadd{APs}}
\newacronym[see={[Glosario:]{JIT}}]{JIT}{JIT}{Just-In-Time \textit{(Justo a Tiempo)} \glsadd{JIT}}
\newacronym[see={[Glosario:]{AOT}}]{AOT}{AOT}{Ahead Of Time \textit{("Adelanto del tiempo")} \glsadd{AOT}}
\newacronym[see={[Glosario:]{RAD}}]{RAD}{RAD}{Rapid Application Development \textit{(Desarrollo Rápido de Aplicaciones)} \glsadd{RAD}}


\newacronym[see={[Glosario:]{PPM}}]{PPM}{PPM}{Portable Pixel Map \textit{(Mapa de Pixeles Portable)}\glsadd{PPM}}
\newacronym[see={[Glosario:]{JPG}}]{JPG}{JPG}{Joint Photographic Experts Group \textit{(Grupo Conjunto de Expertos en Fotografía)}\glsadd{JPG}}