%GLOSARIO


\newglossaryentry{API}
{
	name=API,
	description={una interfaz de programación de aplicaciones (API) es un conjunto particular de reglas (``código'') y las especificaciones que los programas de software pueden seguir para comunicarse entre sí} 
}
\newglossaryentry{borde}
{
	name=borde,
	description={Variaciones fuertes de la intensidad que	corresponden a las fronteras de los objetos visualizados} , plural= bordes
}


\newglossaryentry{chaar}
{
	name=cascada de Haar,
	description={El clasificador Haar es un método desarrollado por Viola y Jones y es una versión del algoritmo Adaboost. Es un clasificador basado en árboles de decisión con entrenamiento supervisado.} ,plural=cascadas de Haar
}
\newglossaryentry{roi}
{
	name=ROI,
	description={\textit{Region Of Interest} Una Región de interés es un subconjunto de una imagen (\gls{2D}) que es usado para un propósito específico} ,plural=ROIs
}
\newglossaryentry{arruga}
{
	name=arruga,
	description={Pliegue que se hace en la piel, ordinariamente por efecto de la edad} ,plural=arrugas
}
\newglossaryentry{mockups}
{
	name=mockups,
	description={Un wireframe o mockup básicamente es un boceto básico y de baja calidad del desarrollo de una página web o el diseño de una interfaz} 
}

\newglossaryentry{tipadodin}
{
	name=tipado dinámico,
	description={A diferencia de lenguajes como C, en lenguajes como \gls{python} a asignación de tipos a las variables se hace en tiempo de ejecución} 
}



\newglossaryentry{degeneracion}
{
	name=degeneración,
	description={facial, se entiende por degeneración cualquier herida, laceración o magulladura} 
}


\newglossaryentry{criminalistica}
{
	name=criminalística,
	description={es una disciplina que aplica fundamentalmente los conocimientos, métodos y técnicas de investigación de las ciencias naturales, en el examen del material sensible significativo relacionado con un presunto hecho delictuoso con el fin de determinar en auxilio de los órganos encargados de administrar justicia} 
}
\newglossaryentry{gui}
{
	name=interfaz de usuario,
	description={la interfaz de usuario es el medio con que el usuario puede comunicarse con una máquina, un equipo o una computadora, y comprende todos los puntos de contacto entre el usuario y el equipo} 
}
\newglossaryentry{riesgo}
{
	name=riesgo,
	description={un riesgo es un problema potencial, puede ocurrir o no} 
}

\newglossaryentry{jython}
{
	name=Jython,
	description={es un lenguaje de programación de alto nivel, dinámico y orientado a objetos basado en Python e implementado íntegramente en Java} 
}



\newglossaryentry{legista}
{
	name=médico legista,
	description={Médico encargado por la justicia para dictaminar los problemas de medicina legal} 
}
\newglossaryentry{orografia}
{
	name=orografía,
	description={Parte de la geografía física que trata de la descripción de las montañas} 
}

\newglossaryentry{arbolDendrtico}
{
	name=arbol dendrítico,
	description={El árbol dendrítico, junto con el pericarion, constituyen partes receptivas de la célula; y son esenciales en la transmisión del impulso nervioso. Esto se debe a que, en respuesta al estímulo por otras células, el potencial de membrana de una célula excitable se despolariza} 
}


\newglossaryentry{neurona}
{
	name=neurona,
	description={son un tipo de células del sistema nervioso cuya principal función es la excitabilidad eléctrica de su membrana plasmática} 
}
\newglossaryentry{3D}
{
	name=3D,
	description={Objeto cuyos componentes pueden ser ubicados mediante una 3-tupla $(x,y,z)$} 
}


\newglossaryentry{2D}
{
	name=2D,
	description={Objeto cuyos componentes pueden ser ubicados mediante una 2-tupla $(x,y)$} 
}

\newglossaryentry{multiplataforma}
{
	name=multiplataforma,
	description={Que puede ser utilizado por distintos sistemas o entornos} 
}

\newglossaryentry{estereosc}
{
	name=imágenes estereoscópicas,
	description={La estereoscopía es cualquier técnica capaz de recoger información visual tridimensional y/o crear la ilusión de profundidad mediante una imagen estereográfica, un estereograma, o una imagen \gls{3D} (tridimensional)} 
}
\newglossaryentry{python}
{
	name=python,
	description={Python es un lenguaje de programación interpretado cuya filosofía hace hincapié en una sintaxis que favorezca un código legible. Se trata de un lenguaje de programación multiparadigma, ya que soporta orientación a objetos, programación imperativa y, en menor medida, programación funcional. Es un lenguaje interpretado, usa tipado dinámico y es multiplataforma} 
} 
\newglossaryentry{bash}
{
	name=bash,
	description={Bash (Bourne again shell) es un programa informático cuya función consiste en interpretar órdenes. Está basado en la shell de Unix y es compatible con POSIX.} 
} 


\newacronym[see={[Glosario:]{NIST}}]{NIST}{NIST}{National Institute of Standards and Technology \textit{(Instituto Nacional de Estándares y Tecnología)}\glsadd{NIST}}
\newacronym[see={[Glosario:]{PPM}}]{PPM}{PPM}{Portable Pixel Map \textit{(Mapa de Pixeles Portable)}\glsadd{PPM}}
\newacronym[see={[Glosario:]{JPG}}]{JPG}{JPG}{Joint Photographic Experts Group \textit{(Grupo Conjunto de Expertos en Fotografía)}\glsadd{JPG}}
\newacronym[see={[Glosario:]{SEMEFO}}]{SEMEFO}{SEMEFO}{ Servicio Médico Forense \glsadd{SEMEFO}}
\newacronym[see={[Glosario:]{IJCF}}]{IJCF}{IJCF}{ Instituto Jaliciense de Ciencias Forenses \glsadd{IJCF}}
\newacronym[see={[Glosario:]{MAD}}]{MAD}{MAD}{ Monto A Depreciar \glsadd{MAD}}
\newacronym[see={[Glosario:]{VDS}}]{VDS}{VDS}{ Valor de Salvamento \glsadd{VDS}}  
\newacronym[see={[Glosario:]{GIF}}]{GIF}{GIF}{Graphics Interchange Format\textit{(Formato de Intercambio de Gráficos)} \glsadd{GIF}} 
\newacronym[see={[Glosario:]{LBPH}}]{LBPH}{LBPH}{Local Binary Patterns Histograms \textit{(Histogramas de Patrones Binarios Locales)} \glsadd{LBPH}}
