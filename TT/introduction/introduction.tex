\newpage 
\chapter*{Introducción}
%\section*{Introducción}
%\blindtext[1]
%\blindmathpaper \blindmathpaper
La socio-físico es una nueva rama interdisciplinaria de la física que aboga por el uso de los métodos y conceptos de la física de sistemas complejos para el estudio de interacciones colectivas en sociedades y de los fenómenos sociales como propiedades emergentes de un conjunto de individuos. Serge Galam propone el uso de estructuras piramidales auto-dirigidas de abajo hacia arriba usando reglas de mayoría. \cite{MarioH.RamirezDiaz2014, Galam.1986, Galam1990, Galam1991, Galam2000}
\\
\\
Con los conceptos que existen relacionados a la socio-física actualmente que cubren varios temas como: redes sociales, evolución de los lenguajes, dinamismo de la población, crecimiento epidemiológico, terrorismo, elecciones electorales entre otros, se desarrollará un sistema computacional enfocado a resolver el problemas de las elecciones electorales relacionados con el éxito o fracaso de estas.
\\
\\
Este documento contendrá el desarrollo del sistema, el primer capítulo presentará el panorama general del sistema, se planteará el problema a resolver y la solución propuesta, el segundo capítulo contendrá los antecedentes de estudios relacionados a la socio-física, el tercer capítulo será un análisis del sistema donde se describirá los datos necesarios para el desarrollo junto con los requerimientos del sistema de acuerdo a la funcionalidad deseada, el cuarto capítulo comprenderá en la descripción del diseño del sistema, se definen los módulos y componentes, el quinto capítulo se detallará las herramientas elegidas para desarrollar el sistema, el sexto capítulo contendrá el desarrollo del sistema y en el séptimo capítulo se presentarán las conclusiones y el trabajo futuro sobre este sistema. 

\subsection*{Origen de la socio-física}
Se plantea que Sege Galam fue uno de los primeros en tratar de publicar sus ideas relacionadas a la socio-física en los años 70's, pero no fue hasta los 90's donde el interés de los físicos creció sobre este campo. Ahora es un campo reconocido de la física emparentado con la física estadística. Este campo ha prosperado y expandido con muchos artículos e investigaciones en revistas importantes de física de todo el mundo, en parte, es gracias a las redes, especialmente a las redes sociales.\cite{Galam.1986} Un ejemplo, fruto de estas investigaciones, es posible cuantificar aproximadamente el grado de amistad usando el tiempo que se habla con otra persona en el celular.\cite{OctavioMiramontes2013}

\subsection*{Física y población}
Existe actualmente una opinión generalizada entre investigadores de las ciencias naturales, donde piensan que el progreso de la humanidad depende primordialmente de avances científicos y tecnológicos los cuales resolverán los mayores problemas de la raza humana, pero esta visión es incompleta. En esta época existen suficientes conocimientos y riqueza para resolver muchos de los dilemas civilizatorios. Si no se aplican es por una combinación de conflictos, intereses e ignorancia. \cite{OctavioMiramontes2013}
\\
\\
La física aplicada al estudio de sistemas sociales puede dividirse en dos ramas: 
\begin{itemize}
    \item La primera plantea el uso de métodos desarrollados por la física para el estudio de sistemas sociales, en donde se pueden encontrar conceptos como: econo-física y la socio-física que modelan aspectos de la economía, organización social, política, antropología, etcétera.
    \item La otra rama consiste en la aplicación de la física mediante la aplicación de sus leyes fundamentales para proveer un marco de restricciones y posibilidades de las sociedades en su conjunto.
\end{itemize}
Ambas ramas dotan de herramientas poderosas para el análisis y la toma de decisiones. \cite{OctavioMiramontes2013}

\subsection*{Física estadística}
La física estadística o mecánica estadística es una rama de la física que se aplica la teoría de probabilidades para determinar el comportamiento de un sistema conformado por partículas. La física estadística proporciona un marco para relacionar las propiedades microscópicas de los átomos y moléculas individuales a las propiedades macroscópicas de los materiales. \cite{Rodriguez2021}

La física estadística permite describir numerosos campos que tengan una naturaleza estocástica como puede ser: reacciones nucleares, sistemas biológicos, químicos, neurológicos, etc.

Esta capacidad de hacer predicciones basadas en las propiedades macroscópicas y microscópicas de los sistemas, permite que el conocimiento obtenido del comportamiento de uno se pueda averiguar detalles del otro.

\subsection*{Econofísica}
El término econofísica fue acuñado a finales del siglo XX, para englobar los trabajos que los físicos hacían sobre diversos aspectos de la economía.
\\
\\
En términos generales, la economía estudia como las personas asignan e intercambian recursos y qué consecuencias tienen estas acciones sobre las decisiones y acciones de otras personas. A grandes rasgos, la economía puede dividirse en:

\begin{itemize}
    \item Microeconomía: Es la rama de la economía que estudia las decisiones de asignación e intercambios de recursos a nivel de agentes individuales y/o firmas.
    \item Macroeconomía: Es la rama de la economía que estudia la economía de una zona, país o grupo de países.
    \item Finanzas: Se relaciona con las decisiones de asignación de recursos en el tiempo bajo condiciones de riesgo como los mercados donde se compran y venden instrumentos financieros como acciones y bonos. Pero también abarca otras áreas, como las finanzas personales, donde las decisiones de asignación de bienes en el tiempo determinan, por ejemplo los fondos de retiro, las inversiones familiares, etc.
\end{itemize}

De hecho, la física estadística tradicionalmente busca ser el puente entre la física microscópica y la fenomenología macroscópica. Donde, desde la perspectiva de la física se ha podido avanzar en el entendimiento de este tipo de sistemas, denominados complejos, en los que, entre otras cosas, pequeñas perturbaciones pueden llevar a efectos enormes, y donde los estados que alcanza el sistema emergen fenómenos colectivos entre los componentes. \cite{Rodriguez2021}

Es relevante denotar que la econofísica se contrapone a la economía clásica en métodos y filosofía ya que, se considera que los fundamentos teóricos derivados de una termodinámica del equilibrio que es inaplicable a la realidad.

\subsection*{La socio-física como una rama de la econofísica}
En el campo de la econofísica toma como agentes los aspectos financieros de alto impacto en los mercados bursátiles, en la bolsa de valores o en cualquier organización como serían las alzas de las acciones o las devaluaciones, la socio-física trabaja de esta manera, analizando una población de agentes, con la diferencia de que la socio-física estudia el comportamiento de individuos poblacionales, para predecir acciones de los mismos, un ejemplo sería la divulgación y el impacto que puede tener un chisme o la ola de los estadios de fútbol.\cite{Galam2008}
% Es posible agregar una imagen de la jerarquía de disciplinas
\thispagestyle{empty}	
 \addcontentsline{toc}{chapter}{Introducción}