\newpage 
\chapter*{Introducción}
%\section*{Introducción}
%\blindtext[1]
%\blindmathpaper \blindmathpaper
La socio-físico es una nueva rama interdisciplinaria de la física que aboga por el uso de los métodos y conceptos de la física de sistemas complejos para el estudio de interacciones colectivas en sociedades y de los fenómenos sociales como propiedades emergentes de un conjunto de individuos. Serge Galam propone el uso de estructuras piramidales auto-dirigidas de abajo hacia arriba usando reglas de mayoría. \cite{MarioH.RamirezDiaz2014, Galam.1986, Galam1990, Galam1991, Galam2000}
\\
\\
Con los conceptos que existen relacionados a la socio-física actualmente que cubren varios temas como: redes sociales, evolución de los lenguajes, dinamismo de la población, crecimiento epidemiológico, terrorismo, elecciones electorales entre otros, se desarrollará un sistema computacional enfocado a resolver el problemas de las elecciones electorales relacionados con el éxito o fracaso de estas.

\thispagestyle{empty}	
 \addcontentsline{toc}{chapter}{Introducción}