\chapter{Nombre de tu sistema (Overview)}
\epigraph{\textit{Beware of bugs in the above code. I've only proved it correct, not tried it.     
	}}{\textit{—  Donald E. Knuth}}
	\vspace*{8cm}
	\begin{center}
		\centering % TU LOGO
		\includegraphics[width=10.5cm]{example-image}
	\end{center}
	\thispagestyle{empty}
	\newpage
	\vspace*{2cm}
\section{Planteamiento del Problema}
Ante la creciente incertidumbre de los partidos políticos sobre la aceptación que tienen en la sociedad, es necesario saber de una manera confiable su nivel de aceptación para que puedan crear sus propuestas para la sociedad.

\section{Solución Propuesta}
Para solventar el problema de la incertidumbre ante el comportamiento de la sociedad, se realizará un sistema de cálculo predictivo usando el enfoque de la socio-física para la predicción electoral y así estimar el grado de aceptación para un partido político determinado y se complementará con el uso de técnicas de algoritmos genéticos para observar la evolución de la población a través del tiempo.

\section{Objetivo}
Desarrollar un sistema computacional basado en el enfoque de socio-físico que permita realizar predicciones de resultados en los procesos electorales. 

\subsection{Objetivos Específicos}
\begin{itemize}
    \item Proporcionar a los políticos un resultado probabilístico confiable sobre el grado de aceptación hacía su partido en una contienda electoral.
    \item Los resultados se compararán con estadísticas reales para comprobar su credibilidad.
\end{itemize}

\section{Justificación}
La aplicación de la teoría de la socio-física y los modelos desarrollados por Serge Galam en el ámbito político de México ayudaría a tener predicciones con el enfoque de la socio-física y no solamente probabilísticos sobre las predicciones electorales con base a los datos recolectados de elecciones anteriores para estimar el grado de aceptación para un partido político determinado.
\\
\\
Para los políticos puede proporcionar resultados estimados sobre la confiabilidad y aceptación en un proceso electoral mediante la socio-física y no solamente probabilísticos.

\section{Estado del Arte}
%Se puede agregar algo, a consultar con los directores o sinodales
\subsection{Herramientas Similares a tu sistema}
\subsubsection{Twitter sentiment analysis for the estimation of voting intention in the 2017 Chilean elections}
Es un sistema basado en el análisis de publicaciones en Twitter basado en técnicas de análisis de sentimientos para realizar una predicción electoral.\cite{TomasAlegreSepulveda2020}

\subsubsection{¿Se pueden predecir geográficamente los resultados electorales? Una aplicación del análisis de clusters y outliers espaciales}
Los resultados de este estudio demuestran que al aplicar la estadística espacial en la geografía electoral es posible predecir los resultados electorales. Se utilizan los conceptos geográficos de cluster y outlier espaciales, y como variable predictiva la segregación espacial socioeconómica.\cite{Perdomo2008}

\subsubsection{Trabajos Terminales}
 
En la tabla \ref{table:similarStuffTTS} se hace mención de sistemas relacionados con \textit{TU SISTEMA}, los cuales han sido desarrollados en las instalaciones de la Escuela Superior de Cómputo.
	
		\begin{table}[H]	
			\centering
 
			  	
			\begin{tabular}{*{2}{|p{6cm}}|p{3cm}|}
				\hline\hline
				
				\cellcolor[gray]{0.8}\textsc{ Software }& \cellcolor[gray]{0.8} \textsc{Características} & \cellcolor[gray]{0.8} \textsc{Precio En El Mercado} \\

				\multicolumn{3}{|c|}{\cellcolor{escom}\textcolor{white}{Trabajos Terminales}}\\
				\hline
				\textit{Identificación de la personalidad de individuos basado en los rasgos faciales con fines criminalísticos (TT 2010 - 0068), ESCOM IPN} & Sistema que trata de realizar la identificación de algun individuo que tenga procesos judiciales. & \textit{\textbf{No Aplica}} \\
				\hline
				\textit{Sistema prototipo para apoyo y seguimiento en procesos judiciales utilizando reconocimiento facial (TT 2004 - 0809), ESCOM IPN} & En este trabajo se trata de realizar la identificación de algun individuo que tenga procesos judiciales para poder hacer un seguimiento del mismo.  & \textit{\textbf{No Aplica}} \\
				\hline
				
				\textit{Envejecimiento de rostro por medio de  \gls{estereosc} (TT 2005 - 0857), ESCOM IPN} & En este trabajo se hace uso de representaciones tridimensionales para poder generar la apariencia posterior de una persona.  &  \textit{\textbf{No Aplica}} \\
				\hline
				\hline
			\end{tabular}
 
		
			\caption{Productos Similares a \textit{``tu sistema''}}
			\label{table:similarStuffTTS}
		\end{table}
		
		
		
\subsection{Lo que ofrece tu sistema}
