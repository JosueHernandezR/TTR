\chapter{Diseño}\label{diseno}

\epigraph{\textit{Sometimes it is the people no one imagines anything of who do the things that no one can imagine.  
}}{\textit{— Alan Turing}}
\vspace*{8cm}
\begin{center}
	\centering
	\includegraphics[width=10.5cm]{example-image}
\end{center}
\thispagestyle{empty}
\newpage
\vspace*{1cm}

En el siguiente capítulo se ve un análisis más a fondo de las reglas de negocio que debe cumplir el sistema, así mismo, el modelo relacional, la arquitectura del sistema usando UML.

\section{Reglas de negocio}
\begin{longtable}{|l|m{4cm}|m{9.5cm}|}
    \hline
    \rowcolor[HTML]{329A9D} 
    \multicolumn{3}{|c|}{\cellcolor[HTML]{329A9D}{\color[HTML]{FFFFFF} Reglas de negocio}} \\ \hline
    \rowcolor[HTML]{3531FF} 
    \multicolumn{1}{|c|}{\cellcolor[HTML]{3531FF}{\color[HTML]{FFFFFF} ID}} & \multicolumn{1}{c|}{\cellcolor[HTML]{3531FF}{\color[HTML]{FFFFFF} Nombre}} & \multicolumn{1}{c|}{\cellcolor[HTML]{3531FF}{\color[HTML]{FFFFFF} Descripción}} \\ \hline
    \endfirsthead
    \hline
    \rowcolor[HTML]{329A9D} 
    \multicolumn{3}{|c|}{\cellcolor[HTML]{329A9D}{\color[HTML]{FFFFFF} Reglas de negocio}} \\ \hline
    \rowcolor[HTML]{3531FF} 
    \multicolumn{1}{|c|}{\cellcolor[HTML]{3531FF}{\color[HTML]{FFFFFF} ID}} & \multicolumn{1}{c|}{\cellcolor[HTML]{3531FF}{\color[HTML]{FFFFFF} Nombre}} & \multicolumn{1}{c|}{\cellcolor[HTML]{3531FF}{\color[HTML]{FFFFFF} Descripción}} \\ \hline
    \endhead
    % aquí añadimos el fondo de todas las hojas, excepto de la última.
    \multicolumn{3}{c}{Sigue en la página siguiente.}
    \endfoot
    % aquí añadimos el fondo de la última hoja.
    \endlastfoot
    RN01&Encuestas&Las encuestas serán limitadas a 10 preguntas. \\ \hline
    RN02&Peso de preguntas & El peso por defecto a cada pregunta creada será de 1.\\ \hline
    RN03&Datos obligatorios & \\ \hline
    RN04&Usuarios identificados & \\ \hline
    RN05&Identificador de usuario único & La dirección de correo electrónico registrada será el identificador de cada usuario, por lo que no podrá haber duplicados. \\ \hline
    \caption{Reglas de negocio}
    \label{table:RNegocio}
\end{longtable}
\section{Casos de uso}
Los requerimientos del sistema se transforman en casos de uso necesarios para plantear la infraestructura del sistema a nivel de implementación, el diagrama de casos de uso se muestra a continuación.

\section{Diagramas de secuencia}
Un diagrama de secuencia muestra la interacción de un conjunto de objetos en una aplicación a través del tiempo y se modela para cada caso de uso.

A continuación se muestran los diagramas de secuencia necesarios por cada caso de uso referentes al módulo de predicción.
\subsection{Realizar predicción}

\subsubsection{Descripción del flujo de información}

\section{Diagramas de estado}

\subsection{Realizar predicción}

\subsubsection{Descripción de estados}

\section{Diagrama de actividades}

\subsection{Realizar predicción}

\section{Diagrama de clases}

\subsection{Tipos de clases}

\subsection{Clases de entidad}

\subsection{Clases de frontera}

\subsection{Clases de control}

\section{Diseño de bases de datos}

\subsection{Diagrama de entidad relación}

\subsection{Modelo relacional}